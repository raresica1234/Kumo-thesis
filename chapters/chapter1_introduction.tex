\chapter{Introduction}

%\chapter*{Introducere}
\label{intro}
% The topic of this thesis is to explore permissions control and security in a real-life 
% example such as a platform for content viewing. In order to facilitate an elaborate enough 
% permission control, Role-Based Access Control (RBAC) will be used. It will also explore .NET, 
% LINQ (SQL operations in .NET), JWT (JSON Web Tokens), React.js and Material UI.

% The primary purpose of this thesis is to provided a well-built application for sharing files
% in a web manner without compromising on security or the privacy of the hosted files. In order 
% to facilitate the security and privacy of the hosted files, the host of the application will 
% use an admin account which has total control over the permissions of its users. The admin 
% can provide default permissions for the default role, specific roles with specific permissions 
% --- for either files, folders or the application's settings and configurations.

\section{What is a permission system?}
To explain what a permission system is, firstly a need to understand what exactly a permission is exists. A permission is more or less a rule which is applied to either an object or an action. A permission system is a way of handling and managing the permissions, as well as putting them into use.
\section{Why are permission systems necessary?}
A permission system is necessary because it let's a user manage individual or group access and handle constraints around various objects, actions, users, or roles. In essence, a permission system has been in use for a very long time. The best example regarding a permission system is a file permission system.
\section{What is the purpose of this thesis?}
The purpose of this thesis is to provide an interactive and more exhaustive system of control in a web environment. The problem that this thesis is trying to address is that most collaborative content viewing solutions such as: Google Drive, or Dropbox explore very simplistic permissions. But because of their simplistic permission approach, a need for a very specific structure for files and folder is needed in other to facilitate a complete control over all of the data a user wants to share. 

Because the needs of a user and what they would want to share with other users will always change, having a very specific structure in mind for the files and folders becomes a big problem that becomes very chaotic and hard to manage. In order to solve this issue, this thesis proposes a more exhaustive permission system for the content of the user, that allows a lot more control.

\section{Thesis structure}
The thesis will contain an additional 5 chapters.
\begin{itemize}
    \item Permission implementation in other apps --- This chapter explores what other applications have done regarding their permission implementation, and why a more exhaustive permission control system might have advantages over the simplistic approaches of other applications.
    \item Permission management --- How exactly are permissions handled and what are the different ways of managing permissions by exploring two major attribute control systems --- ABAC and RBAC.
    \item The application --- This chapter explores how the application actually manages and solves the problem proposed in this thesis. What technologies it uses, and other technical details revolving around the app.
    \item Conclusions and Future work
\end{itemize}
